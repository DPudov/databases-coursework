\Introduction

Визуализация различных явлений становится всё более важной во множестве инженерных
областей знаний.
Задача объёмного рендеринга имеет большое значение, например, в визуализации данных
компьютерной и магнитно-резонансной томографии\cite{book:ash}.
Интерактивная 3D симуляция позволяет учёным ясно воспринимать и
оценивать результаты собственных исследований.

В настоящее время для этого используются различные вычислительные техники обработки
 и графического представления экспериментальных данных\cite{book:physical}.

В данной работе рассматривается 3D симуляция воды. Симуляция жидкостей, в целом,
является примером того, что получаемые о них сведения без соответствующего 3D изображения
довольно сложны для человеческого восприятия.


Целью работы является разработка программного продукта для моделирования движения воды с использованием воксельной графики. Для достижения поставленной цели необходимо решить следующие задачи:

\begin{itemize}
\item проанализировать существующие методы моделирования движения жидкостей и методы рендеринга с помощью вокселей;
\item спроектировать программное обеспечение, симулирующее поведение воды;
\item реализовать программу и проверить её работоспособность.
\end{itemize}
