\chapter{Аналитический раздел}
\label{cha:analysis}
%
% % В начале раздела  можно напомнить его цель
% IDEF0


В данном разделе производится анализ методов вычислений характеристик жидкостей и
их преобразований в графический формат.

Далее рассматриваются идеи применения данных методов к симуляции жидкостей и существующие решения в этой области.

% Обратите внимание, что включается не ../dia/..., а inc/dia/...
% В Makefile есть соответствующее правило для inc/dia/*.pdf, которое
% берет исходные файлы из ../dia в этом случае.

\section{Физическая модель}

Жидкости моделируются как векторное поле скорости жидкости и скалярное поле
плотности. Движение задаётся уравнениями Навье-Стокса \cite{inbook:bigenc}.

Далее рассматривается только движение воды (несжимаемой жидкости) в условиях постоянной температуры.

Тогда уравнения Навье-Стокса в векторной форме принимают следующий вид:
\begin{equation}
    \label{eq:navier-stokes}
    \frac{\partial \vec{v}}{\partial t} + \vec{v} \cdot \nabla\vec{v}= \vec{F} - \frac{1}{\rho} \nabla p + \eta\Delta\vec{v}.
\end{equation}

В уравнении \ref{eq:navier-stokes} $\vec{v}$ - скорость частицы воды,
                                   $t$ - время,
                                   ${\vec{F}}$ - внешняя удельная сила,
                                   $p$ - давление,
                                   $\eta = \frac{\mu}{\rho}$ - кинематический коэффициент вязкости,
                                   $\nabla$ - оператор Гамильтона,
                                   $\Delta$ - оператор Лапласа.

Данная физическая модель лежит в основе многих подходов симуляции жидкостей\cite{book:ash}. Их
обзор приведён далее.

В статистической физике модель поведения частиц жидкости описывается кинетическим
уравнением Больцмана. Данная модель применима для систем, где
есть ограничения на малую скорость частиц\cite{site:bolzman}.

\section{Существующие подходы к симуляции жидкостей}

В вычислениях поведения жидкости необходимо представить физическую модель в
 дискретном виде. Данную проблему решает вычислительная гидродинамика - совокупность
 теоретических, экспериментальных и численных методов, предназначенных для моделирования
 потоковых процессов.

Наиболее распространёнными методами описания характеристик жидкости в
вычислительной гидродинамике являются:
\begin{itemize}
    \item сеточные методы Эйлера;
    \item метод гидродинамики сглаженных частиц;
    \item методы, основанные на турбулентности;
    \item метод решёточных уравнений Больцмана.
\end{itemize}

Сеточные методы Эйлера являются наиболее простым решением симуляции жидкостей
Они заключаются в поиске решения задачи Коши для функций,
заданных таблично. Для каждого узла функции уровня жидкости
 требуется вычисление значений разложений Тейлора в их окрестностям. Решение задачи Коши
 в данном случае аппроксимирует решение уравнения Навье-Стокса\cite{book:compmath}.

Метод гидродинамики сглаженных частиц и методы, основанные на турбулентности,
заключаются в выборе размера частицы ("длины сглаживания"), на котором их свойства
"сглаживаются" посредством функции ядра или интерполяции, и решения уравнений
Навье-Стокса с учётом вязкости и плотности. Это позволяет эффективно моделировать
поведение жидкостей, газов и даже использовать в астрофизике\cite{site:astro}.

 Метод решёточных уравнений Больцмана основан на кинетическом уравнении Больцмана,
 упомянутом ранее. Этот метод поддерживает многофазные жидкости, наличие теплопроводности
 и граничные условия на макроскопическом уровне\cite{site:habr-physics}.

\section{Анализ методов визуализации жидкостей}

Основными требованиями к методам симуляции жидкостей со стороны компьютерной графики
являются визуальная правдоподобность и скорость анимации.

% вывод по разделу

\section{Вывод}


%%% Local Variables:
%%% mode: latex
%%% TeX-master: "rpz"
%%% End:
