%% Преамбула TeX-файла

% 1. Стиль и язык
\documentclass[utf8x, 14pt]{G7-32} % Стиль (по умолчанию будет 14pt)

% Остальные стандартные настройки убраны в preamble.inc.tex.
\sloppy

% Настройки стиля ГОСТ 7-32
% Для начала определяем, хотим мы или нет, чтобы рисунки и таблицы нумеровались в пределах раздела, или нам нужна сквозная нумерация.
\EqInChapter % формулы будут нумероваться в пределах раздела
\TableInChapter % таблицы будут нумероваться в пределах раздела
\PicInChapter % рисунки будут нумероваться в пределах раздела

% Добавляем гипертекстовое оглавление в PDF
\usepackage[
bookmarks=true, colorlinks=true, unicode=true,
urlcolor=black,linkcolor=black, anchorcolor=black,
citecolor=black, menucolor=black, filecolor=black,
]{hyperref}

\AfterHyperrefFix

\usepackage{microtype}% полезный пакет для микротипографии, увы под xelatex мало чего умеет, но под pdflatex хорошо улучшает читаемость

% Тире могут быть невидимы в Adobe Reader
\ifInvisibleDashes
\MakeDashesBold
\fi

\usepackage{graphicx}   % Пакет для включения рисунков

% С такими оно полями оно работает по-умолчанию:
% \RequirePackage[left=20mm,right=10mm,top=20mm,bottom=20mm,headsep=0pt,includefoot]{geometry}
% Если вас тошнит от поля в 10мм --- увеличивайте до 20-ти, ну и про переплёт не забывайте:
\geometry{right=20mm}
\geometry{left=30mm}
\geometry{bottom=20mm}
\geometry{ignorefoot}% считать от нижней границы текста


% Пакет Tikz
\usepackage{tikz}
\usetikzlibrary{arrows,positioning,shadows}

\usepackage{epstopdf}
% Произвольная нумерация списков.
\usepackage{enumerate}

% ячейки в несколько строчек
\usepackage{multirow}

% itemize внутри tabular
\usepackage{paralist,array}

%\setlength{\parskip}{1ex plus0.5ex minus0.5ex} % разрыв между абзацами
\setlength{\parskip}{1ex} % разрыв между абзацами
\usepackage{blindtext}

% Центрирование подписей к плавающим окружениям
%\usepackage[justification=centering]{caption}

\usepackage{newfloat}
\DeclareFloatingEnvironment[
placement={!ht},
name=Equation
]{eqndescNoIndent}
\edef\fixEqndesc{\noexpand\setlength{\noexpand\parindent}{\the\parindent}\noexpand\setlength{\noexpand\parskip}{\the\parskip}}
\newenvironment{eqndesc}[1][!ht]{%
    \begin{eqndescNoIndent}[#1]%
\fixEqndesc%
}
{\end{eqndescNoIndent}}


% Настройки листингов.
\ifPDFTeX
% 8 Листинги

\usepackage{listings}

% Значения по умолчанию
\lstset{
  basicstyle= \footnotesize,
  breakatwhitespace=true,% разрыв строк только на whitespacce
  breaklines=true,       % переносить длинные строки
%   captionpos=b,          % подписи снизу -- вроде не надо
  inputencoding=koi8-r,
  numbers=left,          % нумерация слева
  numberstyle=\footnotesize,
  showspaces=false,      % показывать пробелы подчеркиваниями -- идиотизм 70-х годов
  showstringspaces=false,
  showtabs=false,        % и табы тоже
  stepnumber=1,
  tabsize=4,              % кому нужны табы по 8 символов?
  frame=single
}

% Стиль для псевдокода: строчки обычно короткие, поэтому размер шрифта побольше
\lstdefinestyle{pseudocode}{
  basicstyle=\small,
  keywordstyle=\color{black}\bfseries\underbar,
  language=Pseudocode,
  numberstyle=\footnotesize,
  commentstyle=\footnotesize\it
}

% Стиль для обычного кода: маленький шрифт
\lstdefinestyle{realcode}{
  basicstyle=\scriptsize,
  numberstyle=\footnotesize
}

% Стиль для коротких кусков обычного кода: средний шрифт
\lstdefinestyle{simplecode}{
  basicstyle=\footnotesize,
  numberstyle=\footnotesize
}

% Стиль для BNF
\lstdefinestyle{grammar}{
  basicstyle=\footnotesize,
  numberstyle=\footnotesize,
  stringstyle=\bfseries\ttfamily,
  language=BNF
}

% Определим свой язык для написания псевдокодов на основе Python
\lstdefinelanguage[]{Pseudocode}[]{Python}{
  morekeywords={each,empty,wait,do},% ключевые слова добавлять сюда
  morecomment=[s]{\{}{\}},% комменты {а-ля Pascal} смотрятся нагляднее
  literate=% а сюда добавлять операторы, которые хотите отображать как мат. символы
    {->}{\ensuremath{$\rightarrow$}~}2%
    {<-}{\ensuremath{$\leftarrow$}~}2%
    {:=}{\ensuremath{$\leftarrow$}~}2%
    {<--}{\ensuremath{$\Longleftarrow$}~}2%
}[keywords,comments]

% Свой язык для задания грамматик в BNF
\lstdefinelanguage[]{BNF}[]{}{
  morekeywords={},
  morecomment=[s]{@}{@},
  morestring=[b]",%
  literate=%
    {->}{\ensuremath{$\rightarrow$}~}2%
    {*}{\ensuremath{$^*$}~}2%
    {+}{\ensuremath{$^+$}~}2%
    {|}{\ensuremath{$|$}~}2%
}[keywords,comments,strings]

% Подписи к листингам на русском языке.
\renewcommand\lstlistingname{Листинг}
\renewcommand\lstlistlistingname{Листинги}

\else
\usepackage{local-minted}
\fi

% Полезные макросы листингов.
% Любимые команды
\newcommand{\Code}[1]{\textbf{#1}}


% Стиль титульного листа и заголовки

%\NirEkz{Экз. 3}                                  % Раскоментировать если не требуется
%\NirGrif{Секретно}                % Наименование грифа

%\gosttitle{Gost7-32}       % Шаблон титульной страницы, по умолчанию будет ГОСТ 7.32-2001,
% Варианты GostRV15-110 или Gost7-32

\NirOrgLongName{Министерство образования и науки
Российской Федерации\\
Федеральное государственное бюджетное образовательное учреждение высшего образования
\par
МОСКОВСКИЙ ГОСУДАРСТВЕННЫЙ ТЕХНИЧЕСКИЙ УНИВЕРСИТЕТ ИМ. Н. Э. БАУМАНА (национальный исследовательский университет)
}                                           %% Полное название организации

% \NirUdk{УДК № 378.14}
% \NirGosNo{№ госрегистрации 01970006723}
%\NirInventarNo{Инв. № ??????}

%\NirConfirm{Согласовано}                  % Смена УТВЕРЖДАЮ
% \NirBoss[.49]{Проректор университета\\по научной работе}{Н.С. Жернаков}            %% Заказчик, утверждающий НИР


%\NirReportName{Научно-технический отчет}   % Можно поменять тип отчета
%\NirAbout{О составной части \par опытно-конструкторской работы} %Можно изменить о чем отчет

%\NirPartNum{Часть}{1}                      % Часть номер

%\NirBareSubject{}                  % Убирает по теме если раскоментить

% \NirIsAnnotacion{АННОТАЦИОННЫЙ }         %% Раскомментируйте, если это аннотационный отчёт
%\NirStage{промежуточный}{Этап \No 1}{} %%% Этап НИР: {номер этапа}{вид отчёта - промежуточный или заключительный}{название этапа}
%\NirStage{}{}{} %%% Этап НИР: {номер этапа}{вид отчёта - промежуточный или

% \Nir{}

\NirSubject{Программа моделирования движения воды с использованием вокселов }                                   % Наименование темы
%\NirFinal{}                        % Заключительный, если закоментировать то промежуточный
%\finalname{итоговый}               % Название финального отчета (Заключительный)
%\NirCode{Шифр\,---\,САПР-РЛС-ФИЗТЕХ-1} % Можно задать шифр как в ГОСТ 15.110
\NirCode{}

% \NirManager{Зам. проректора по научной работе}{Р.А. Бадамшин  } %% Название руководителя
\NirIsp{Руководитель темы}{А. С. Кострицкий} %% Название руководителя

\NirYear{2019}%% если нужно поменять год отчёта; если закомментировано, ставится текущий год
\NirTown{Москва}                           %% город, в котором написан отчёт



\begin{document}

\frontmatter % выключает нумерацию ВСЕГО; здесь начинаются ненумерованные главы: реферат, введение, глоссарий, сокращения и прочее.

\maketitle %создает титульную страницу

%
% \begin{executors}
% \personalSignature{Первый исполнитель}{ФИО}
%
% \personalSignature{Второй исполнитель}{ФИО}
% \end{executors}


%\listoffigures                         % Список рисунков

%\listoftables                          % Список таблиц

%\NormRefs % Нормативные ссылки
% Команды \breakingbeforechapters и \nonbreakingbeforechapters
% управляют разрывом страницы перед главами.
% По-умолчанию страница разрывается.

% \nobreakingbeforechapters
% \breakingbeforechapters

% Также можно использовать \Referat, как в оригинале
\begin{abstract}

    Отчет содержит \pageref{LastPage}\,стр.%
    \ifnum \totfig >0
    , \totfig~рис.%
    \fi
    \ifnum \tottab >0
    , \tottab~табл.%
    \fi
    %
    \ifnum \totbib >0
    , \totbib~источн.%
    \fi
    %
    \ifnum \totapp >0
    , \totapp~прил.%
    \else
    .%
    \fi


    Это пример каркаса расчётно-пояснительной записки, желательный к использованию в РПЗ проекта по курсу РСОИ
    \nocite{*}.

\end{abstract}

%%% Local Variables:
%%% mode: latex
%%% TeX-master: "rpz"
%%% End:


\tableofcontents

\printnomenclature % Автоматический список сокращений

\Introduction

Визуализация различных явлений становится всё более важной во множестве инженерных
областей знаний.
Задача объёмного рендеринга имеет большое значение, например, в визуализации данных
компьютерной и магнитно-резонансной томографии\cite{book:ash}.
Интерактивная 3D симуляция позволяет учёным ясно воспринимать и
оценивать результаты собственных исследований.

В настоящее время для этого используются различные вычислительные техники обработки
 и графического представления экспериментальных данных\cite{book:physical}.

В данной работе рассматривается 3D симуляция воды. Симуляция жидкостей, в целом,
является примером того, что получаемые о них сведения без соответствующего 3D изображения
довольно сложны для человеческого восприятия.


Целью работы является разработка программного продукта для моделирования движения воды с использованием воксельной графики. Для достижения поставленной цели необходимо решить следующие задачи:

\begin{itemize}
\item проанализировать существующие методы моделирования движения жидкостей и методы рендеринга с помощью вокселей;
\item спроектировать программное обеспечение, симулирующее поведение воды;
\item реализовать программу и проверить её работоспособность.
\end{itemize}


\mainmatter % это включает нумерацию глав и секций в документе ниже

\chapter{Аналитический раздел}
\label{cha:analysis}
%
% % В начале раздела  можно напомнить его цель
% IDEF0


В данном разделе производится анализ методов вычислений характеристик жидкостей и
их преобразований в графический формат.

Далее рассматриваются идеи применения данных методов к симуляции жидкостей и существующие решения в этой области.

% Обратите внимание, что включается не ../dia/..., а inc/dia/...
% В Makefile есть соответствующее правило для inc/dia/*.pdf, которое
% берет исходные файлы из ../dia в этом случае.

\section{Физическая модель}

Жидкости моделируются как векторное поле скорости жидкости и скалярное поле
плотности. Движение задаётся уравнениями Навье-Стокса \cite{inbook:bigenc}.

Далее рассматривается только движение воды (несжимаемой жидкости) в условиях постоянной температуры.

Тогда уравнения Навье-Стокса в векторной форме принимают следующий вид:
\begin{equation}
    \label{eq:navier-stokes}
    \frac{\partial \vec{v}}{\partial t} + \vec{v} \cdot \nabla\vec{v}= \vec{F} - \frac{1}{\rho} \nabla p + \eta\Delta\vec{v}.
\end{equation}

В уравнении \ref{eq:navier-stokes} $\vec{v}$ - скорость частицы воды,
                                   $t$ - время,
                                   ${\vec{F}}$ - внешняя удельная сила,
                                   $p$ - давление,
                                   $\eta = \frac{\mu}{\rho}$ - кинематический коэффициент вязкости,
                                   $\nabla$ - оператор Гамильтона,
                                   $\Delta$ - оператор Лапласа.

Данная физическая модель лежит в основе многих подходов симуляции жидкостей\cite{book:ash}. Их
обзор приведён далее.

В статистической физике модель поведения частиц жидкости описывается кинетическим
уравнением Больцмана. Данная модель применима для систем, где
есть ограничения на малую скорость частиц\cite{site:bolzman}.

\section{Существующие подходы к симуляции жидкостей}

В вычислениях поведения жидкости необходимо представить физическую модель в
 дискретном виде. Данную проблему решает вычислительная гидродинамика - совокупность
 теоретических, экспериментальных и численных методов, предназначенных для моделирования
 потоковых процессов.

Наиболее распространёнными методами описания характеристик жидкости в
вычислительной гидродинамике являются:
\begin{itemize}
    \item сеточные методы Эйлера;
    \item метод гидродинамики сглаженных частиц;
    \item методы, основанные на турбулентности;
    \item метод решёточных уравнений Больцмана.
\end{itemize}

Сеточные методы Эйлера являются наиболее простым решением симуляции жидкостей
Они заключаются в поиске решения задачи Коши для функций,
заданных таблично. Для каждого узла функции уровня жидкости
 требуется вычисление значений разложений Тейлора в их окрестностям. Решение задачи Коши
 в данном случае аппроксимирует решение уравнения Навье-Стокса\cite{book:compmath}.

Метод гидродинамики сглаженных частиц и методы, основанные на турбулентности,
заключаются в выборе размера частицы ("длины сглаживания"), на котором их свойства
"сглаживаются" посредством функции ядра или интерполяции, и решения уравнений
Навье-Стокса с учётом вязкости и плотности. Это позволяет эффективно моделировать
поведение жидкостей, газов и даже использовать в астрофизике\cite{site:astro}.

 Метод решёточных уравнений Больцмана основан на кинетическом уравнении Больцмана,
 упомянутом ранее. Этот метод поддерживает многофазные жидкости, наличие теплопроводности
 и граничные условия на макроскопическом уровне\cite{site:habr-physics}.

\section{Анализ методов визуализации жидкостей}

Основными требованиями к методам симуляции жидкостей со стороны компьютерной графики
являются визуальная правдоподобность и скорость анимации.

% вывод по разделу

\section{Вывод}


%%% Local Variables:
%%% mode: latex
%%% TeX-master: "rpz"
%%% End:

\chapter{Конструкторский раздел}
\label{cha:design}

В данном разделе описано проектирование метода рендеринга воды с помощью вокселов
с указанием соответствующих схем алгоритмов.
% схемы алгоритмов, расписать IDEF0, спускаясь к деталям каждого этапа

\section{Архитектура приложения}

% Вывод по разделу
\section{Вывод}


%%% Local Variables:
%%% mode: latex
%%% TeX-master: "rpz"
%%% End:

\chapter{Технологический раздел}
\label{cha:impl}

В данном разделе описаны требуемые средства и подходы к реализации ПО по ранее указанным методам.

% самые важные листинги
% технологические куски
% изящные вещи
% выбор средств реализации: ЯП, ОС, библиотеки
% readme, инструкции запуска-удаления и т.п.

\section{Требования к программному обеспечению}

Разработанное ПО должно моделировать движение воды с использованием вокселов.

Пользователь должен иметь возможность изменять степень детализации,
источники освещения, выбирать вид отображения воды:
\begin{itemize}
    \item волны;
    \item спокойное течение.
\end{itemize}

Моделирование движения должно осуществляться с использованием операций переноса, масштабирования и поворота.

% вывод по разделу
\section{Вывод}

%%% Local Variables:
%%% mode: latex
%%% TeX-master: "rpz"
%%% End:

\chapter{Исследовательский раздел}
\label{cha:research}

В данном разделе проводится апробация разработанной программы.
% скорость работы от параметров рендеринга
% другие метрики
% попробовать распараллелить

% листочки с презентацией подготовить
% демо программы


\section{Примеры использования}

\section{Выводы}



%%% Local Variables:
%%% mode: latex
%%% TeX-master: "rpz"
%%% End:

% \chapter{Организационно-экономический раздел}
\label{cha:econom}

\section{Протестируем специальные символы.}

И заодно переключение шрифтов.


{\shorthandoff" \texttt{"-{}-* Прямая речь "-{}-{}- <{}<после ,{},тире`{}` неразрывный пробел>{}>}}

{\cyrillicfonttt{\bfseries\itshape\textbackslash{}cyrillicfonttt}
"--* Прямая речь "--- <<после ,,тире`` неразрывный пробел>>.}

{\cyrillicfontsf{\bfseries\itshape\textbackslash{}cyrillicfontsf}
"--* Прямая речь "--- <<после ,,тире`` неразрывный пробел>>.}

{\cyrillicfont{\bfseries\itshape\textbackslash{}cyrillicfont}
"--* Прямая речь "--- <<после ,,тире`` неразрывный пробел>>.}


\blindtext
%%% Local Variables:
%%% mode: latex
%%% TeX-master: "rpz"
%%% End:

% \chapter{Промышленная экология и безопасность}\label{cha:bzd}

\blindtext

\blindlistlist[3]{enumerate}

%%% Local Variables:
%%% mode: latex
%%% TeX-master: "rpz"
%%% End:


\backmatter %% Здесь заканчивается нумерованная часть документа и начинаются ссылки и

\Conclusion % заключение к отчёту

%цели были достигнуты, были выбраны такие-то алгоритмы, я молодец
В результате проделанной работы стало ясно, что ничего не ясно...

%%% Local Variables:
%%% mode: latex
%%% TeX-master: "rpz"
%%% End:
%% заключение


% % Список литературы при помощи BibTeX
% Юзать так:
%
% pdflatex rpz
% bibtex rpz
% pdflatex rpz

\bibliographystyle{ugost2008}
\bibliography{rpz}

%%% Local Variables: 
%%% mode: latex
%%% TeX-master: "rpz"
%%% End: 



\appendix   % Тут идут приложения

\chapter{Картинки}
\label{cha:appendix1}

\blindtext
\begin{figure}
\centering
\caption{Картинка в приложении. Страшная и ужасная.}
\end{figure}

%%% Local Variables: 
%%% mode: latex
%%% TeX-master: "rpz"
%%% End: 


\chapter{Еще картинки}
\label{cha:appendix2}
\blindtext

\begin{figure}
\centering
\caption{Еще одна картинка, ничем не лучше предыдущей. Но надо же как-то заполнить место.}
\end{figure}

%%% Local Variables: 
%%% mode: latex
%%% TeX-master: "rpz"
%%% End: 


\end{document}

%%% Local Variables:
%%% mode: latex
%%% TeX-master: t
%%% End:
